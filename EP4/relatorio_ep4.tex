\documentclass[a4paper,11pt]{article}
\setlength{\topmargin}{-.5in}
\setlength{\textheight}{5in}
\setlength{\textwidth}{6.3in}
\setlength{\oddsidemargin}{-.125in}
\setlength{\evensidemargin}{-.125in}

\usepackage[T1]{fontenc}
\usepackage[portuguese]{babel}
\usepackage{csquotes}
\usepackage{amsmath}
\usepackage{amsfonts}
\usepackage{amssymb}
\usepackage{float}
\usepackage{geometry}
\geometry{includeheadfoot,margin=1.9cm}
\usepackage{algorithm}
\usepackage{algorithmic}

\title{Exercício Programa 4 - Métodos iterativos para sistemas lineares: Gradientes Conjugados}

\author{
  Daniela Gonzalez Favero - 10277443
}
\date{13 de Dezembro de 2020}

\begin{document}

    \maketitle
    
    \section{O Método de Gradientes Conjugados}
        Esta primeira parte do relatório tem como objetivo fazer um resumo descrevendo o método de Gradientes Conjugados. Para isto, usei como base o capítulo 38 do livro \textit{"Numerical Linear Algebra"\ – Lloyd N. Trefethen, David Bau (Society for Industrial and Applied Mathematics, Philadelphia, 1997)}.
    
        \subsection{Definindo o problema}
            Seja $A \in \mathbb{R}^{m \times m}$ simétrica e definida positiva, queremos resolver o sistema de equações não-singular $Ax=b$, com solução exata $x_* = A^{-1}b$. Vamos usar $\mathcal{K}_n$ para denotar o n-ésimo subespaço de Krylov gerado por $b$: $\mathcal{K}_n = \langle b,Ab, \dots , A^{n-1}b \rangle$.
            
            A função $\| \cdot \|_A$ definida por $\| x \|_A = \sqrt{x^TAx}$ é a $A$-norma sobre $\mathbb{R}^{m}$. O vetor cuja $A$-norma nos interessa é $e_n=x_*-x_n$, o erro no passo $n$. A iteração do gradiente conjugado se dá por um sistema de fórmulas de recorrência que gera uma sequência única de iterados $\{ x_n \in \mathcal{K}_n \}$, com a propriedade de que no passo $n$, $\| e_n \|_A$ é minimizado.
        \subsection{O algoritmo}
            O esqueleto do algoritmo se dá por:
            \begin{algorithm}
                \renewcommand\thealgorithm{}
                \caption{Gradientes Conjugados}
                \begin{algorithmic} 
                    \STATE $x_0 \leftarrow 0,\ r_0 \leftarrow b,\ p_0 \leftarrow r_0$
                    \FOR{n\ =\ 1,2,3,\dots}
                        \STATE $\alpha_n \leftarrow (r^T_{n-1}r_{n-1})/(p^T_{n-1}Ap_{n-1})$
                        \STATE $x_n \leftarrow x_{n-1} + \alpha_np_{n-1}$
                        \STATE $r_n \leftarrow r_{n-1} - \alpha_nAp_{n-1}$
                        \STATE $\beta_n \leftarrow (r^T_nr_n)/(r^T_{n-1}r_{n-1})$
                        \STATE $p_n \leftarrow r_n + \beta_np_{n-1}$
                    \ENDFOR
                \end{algorithmic}
            \end{algorithm}
            
            A cada passo, o algoritmo realiza diversas manipulações com vetores e um produto matriz-vetor ($Ap_{n-1}$). Se $A$ for densa e não-estruturada, o produto matriz-vetor domina a contagem de operações, de modo que cada passo leva aproximadamente $2m^2$ flops. Caso contrário, $Ap_{n-1}$ pode ser computado em $\mathcal{O}(m)$ operações.
        
        \subsection{Propriedades do método}
            A partir do algoritmo podemos deduzir algumas propriedades, dentre elas: \\
            \textbf{Teorema:} Tome a aplicação do algoritmo de gradientes conjugados em um problema $Ax=b$, com $A$ simétrica e definida positiva. Enquanto não converge, o algoritmo procede sem divisões por zero, e temos as seguintes identidades dos subespaços:
            $$
                \mathcal{K}_n = \langle x_1, x_2, \dots, x_n  \rangle = \langle p_0, p_1, \dots, p_{n-1}  \rangle = \langle r_0, r_1, \dots, r_{n-1}  \rangle = \langle b, Ab, \dots, A^{n-1}b  \rangle.
            $$
            Além disso, os resíduos são ortogonais:
            $$
                r^T_nr_j = 0\ \ (j<n),
            $$
            e as direções de busca são "$A$-conjugadas":
            $$
                p^T_nAp_j = 0\ \ (j<n).
            $$
            
            Desse teorema, segue um corolário sobre a minimalidade de $\| e_n \|_A$: \\
            \textbf{Corolário:}  Tome a aplicação do algoritmo de gradientes conjugados em um problema $Ax=b$, com $A$ simétrica e definida positiva. Se o algoritmo ainda não convergiu, então $x_n$ é o único ponto em $\mathcal{K}_n$ que minimiza $\| e_n \|_A$. A convergência é monótona,
            $$
                \| e_n \|_A \leq \| e_{n-1} \|_A,
            $$
            e $e_n = 0$ é atingido para algum $n \leq m$.
        
        \subsection{Outras aplicações}
            É importante notar que a propriedade de otimalidade do algoritmo de gradientes conjugados de minimizar  $\| e_n \|_A$ no passo $n$ sobre todos os vetores $x \in \mathcal{K}_n$ mostra que este é um algoritmo de otimização. Com ele, pode-se minimizar a função quadrática $\varphi(x) =  \frac{1}{2}x^TAx-x^Tb$, dados $A$ e $b$ e $x \in \mathbb{R}^m$. A cada passo, uma iteração $x_n = x_{n-1}+ \alpha_np_{n-1}$ é computada de tal forma que minimize $\varphi(x)$ sobre todo $x$ no espaço unidimensional $x_{n-1}+ \langle p_{n-1} \rangle$.
            
            O método de gradientes conjugados também pode ser utilizado para aproximação de polinômios, por meio da $A$-norma do erro. O problema pode ser descrito como: Encontrar $p_n \in P_n$ tal que $\| p_n (A) e_0 \|_A =$ mínimo, sendo $e_0=x_*-x_0=x_*$ o erro inicial. A convergência do método decorre do seguinte teorema:\\
            \textbf{Teorema:} Se o método de gradientes conjugados ainda não convergiu antes do passo $n$, então o problema da aproximação de polinômios tem uma solução única $p_n \in P_n$, e o iterado $x_n$ tem erro $e_n=p_n(A)e_0$ para esse mesmo polinômio $p_n$. Consequentemente, temos:
             $$
                \frac{\|e_n\|_A}{\|e_0\|_A} = \inf_{p \in P_n} \frac{\|p(A)e_0\|_A}{\|e_0\|_A} \leq \inf_{p \in P_n} \max_{\lambda \in \Lambda(A)} |p(\lambda)|,
             $$
             onde $\Lambda(A)$ denota o espectro de A.
            
        \subsection{Taxa de convergência}
            O último teorema acima estabelece que a taxa de convergência do método de gradientes conjugados é determinado pela localização do espectro de $A$. Um bom espectro é o tem seus polinômios $p_n \in P_n$ podendo ser bem pequenos, com o tamanho diminuindo rapidamente com $n$. Dois corolários desse teorema explicam como isso pode acontecer:\\
            \textbf{Corolário:} Se $A$ tem apenas $n$ autovalores distintos, então o método de gradientes conjugados converge em no máximo $n$ passos.
            
            No outro extremo:\\
            \textbf{Corolário:} Tome o método de gradientes conjugados aplicado a um problema $Ax=b$, com $A$ simétrica, definida positiva e cujo número de condição 2-norma é $\kappa$. Então as $A$-normas dos erros satisfazem
            $$
                \frac{\|e_n\|_A}{\|e_0\|_A} \leq 2 \left/ \left[ \left( \frac{\sqrt{\kappa}+1}{\sqrt{\kappa}-1} \right)^n + \left( \frac{\sqrt{\kappa}+1}{\sqrt{\kappa}-1} \right)^{-n} \right] \leq 2 \left( \frac{\sqrt{\kappa}-1}{\sqrt{\kappa}+1} \right)^n.
            $$
            
            Este segundo corolário é o mais famoso sobre a convergência do método de gradientes conjugados. Como
            $$
                \frac{\sqrt{\kappa}-1}{\sqrt{\kappa}+1} ~ 1 - \frac{2}{\sqrt{\kappa}}
            $$
            quando $\kappa \to \infty$, isso implica que se $\kappa$ é grande mas não muito grande, a convergência para uma tolerância especificada pode ser de $\mathcal{O}(\sqrt{\kappa})$ iterações.

    \section{Experimentos}
        \subsection{Gerador de testes}
            Usando a ideia do exemplo no capítulo 38 do livro \textit{"Numerical Linear Algebra"}, fiz um programa em \textit{FORTRAN} que gera matrizes positivas definidas, simétricas e esparsas. O código que está em \texttt{generator.f90} recebe a dimensão \texttt{m} do sistema a ser gerado aleatoriamente e o parâmetro $\tau$ que controla o quão esparsa será a matriz $A$. O programa imprime, na primeira linha, o valor \texttt{m}; nas $m^2$ linhas seguintes, dois inteiros $i,j$ e um real $a_{ij}$, tal que $A(i,j)=a_{ij}$; e nas últimas $m$ linhas imprime um inteiro $i$ e um real $b_i$, tal que $b(i)=b_i$. Esse é o formato de entrada que o arquivo \texttt{ep4.f90} com o método de gradientes conjugados (e também o arquivo \texttt{ep2.f90} entregue anteriormente com a decomposição de Cholesky) aceita.
        
        \subsection{Comparando o consumo de tempo}    
            Utilizando as mesmas entradas com a decomposição de Cholesky (código feito no \textbf{EP2}) e variando o valor de $\tau$, foi possível comparar o consumo de tempo (em segundos) dos algoritmos:
            \begin{center}
                \begin{tabular}{ | c | c | c | c | } 
                    \hline
                    & & \\ [-1em]
                    $m$ & $\tau$ & Cholesky & Gradientes Conjugados\\  [+.5em]
                    \hline\hline
                    & & & \\ [-1em]
                    500 & 0.01 & $ 6.37\times 10^{-2}$ & $ 2.93 \times 10^{-3}$ \\ [+.5em]
                    \hline
                    & & & \\ [-1em]
                    500 & 0.05 & $ 6.37 \times 10^{-2}$ & $ 6.33 \times 10^{-3}$ \\ [+.5em]
                    \hline
                    & & & \\ [-1em]
                    500 & 0.1 & $ 6.37 \times 10^{-2}$ & $ 2.02 \times 10^{-2}$ \\ [+.5em]
                    \hline
                    & & & \\ [-1em]
                    500 & 0.2 & $ 6.20 \times 10^{-2}$ & $ 4.50 \times 10^{-1}$ \\ [+.5em]
                    \hline\hline
                    & & & \\ [-1em]
                    1000 & 0.01 & $ 5.02 \times 10^{-1}$ & $ 1.43 \times 10^{-2}$ \\ [+.5em]
                    \hline
                    & & & \\ [-1em]
                    1000 & 0.05 & $ 5.10 \times 10^{-1}$ & $ 3.32 \times 10^{-2}$ \\ [+.5em]
                    \hline
                    & & & \\ [-1em]
                    1000 & 0.1 & $ 4.81 \times 10^{-1}$ & $ 1.19 $ \\ [+.5em]
                    \hline
                    & & & \\ [-1em]
                    1000 & 0.2 & $ 5.02 \times 10^{-1}$ & $ 5.21 $ \\ [+.5em]
                    \hline
                \end{tabular}
            \end{center}
            A partir da tabela acima, é possível concluir que conforme mais esparsa (principalmente com $\tau = 0.01$) fica a matriz, mais compensa utilizar o método de gradientes conjugados. No entanto, quando a matriz não é esparsa ($\tau = 0.2$), o método de Cholesky é melhor. 

\end{document}
