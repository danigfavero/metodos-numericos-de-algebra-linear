\documentclass[a4paper,11pt]{article}
\usepackage{dmasproject}
\usepackage{multirow}


\title{Exercício Programa 2 - Resolução de sistemas de equações lineares}

\author{
  Daniela Gonzalez Favero - 10277443
}
\date{11 de Outubro de 2020}

\begin{document}

\maketitle


\section{Primeira parte:  sistemas definidos positivos}

Tabela:

\begin{table}[h]
\begin{tabular}{|c|c|c|c|c|c|c|}
\hline
\multirow{3}{*}{\textbf{PROBLEMA}} & \multicolumn{6}{c|}{\textbf{DECOMPOSIÇÃO DE CHOLESKY}}                                             \\ \cline{2-7} 
                                   & \multicolumn{3}{c|}{\textbf{ORIENTADA A LINHA}} & \multicolumn{3}{c|}{\textbf{ORIENTADA A COLUNA}} \\ \cline{2-7} 
                                   & \textbf{$A=GG^T$}      & \textbf{$Gy=b$}      & \textbf{$G^Tx=y$}     & \textbf{$A=GG^T$}      & \textbf{$Gy=b$}      & \textbf{$G^Tx=y$}      \\ \hline
\textbf{1}                         &  0              &  0              &  0             &  0              &   0             &      0          \\ \hline
\textbf{2}                         &   0             &     0           &  0             &    0            &     0           &       0         \\ \hline
\textbf{3}                         &           0     &         0       &        0       &       0         &       0         &       0         \\ \hline
\textbf{4}                         &         0       &        0        &      0         &        0        &       0         &      0          \\ \hline
\textbf{5}                         &        0        &      0          &        0       &         0       &        0        &      0          \\ \hline
\textbf{6}                         &         0       &       0         &         0      &     0           &       0         &      0          \\ \hline
\textbf{7}                         &         0       &         0       &         0      &      0          &         0       &       0         \\ \hline
\end{tabular}
\end{table}

Comentário sobre a tabela.


\section{Segunda parte:  sistemas gerais}

Tabela:

% Please add the following required packages to your document preamble:
% \usepackage{multirow}
\begin{table}[h]
\begin{tabular}{|c|c|c|c|c|}
\hline
\multirow{3}{*}{\textbf{PROBLEMA}} & \multicolumn{4}{c|}{\textbf{DECOMPOSIÇÃO DE CHOLESKY}}                                             \\ \cline{2-5} 
                                   & \multicolumn{2}{c|}{\textbf{ORIENTADA A LINHA}} & \multicolumn{2}{c|}{\textbf{ORIENTADA A COLUNA}} \\ \cline{2-5} 
                                   & \textbf{$PA=LU$}         & \textbf{$LUx=Pb$}        & \textbf{$PA=LU$}         & \textbf{$LUx=Pb$}         \\ \hline
\textbf{1}                         & 0                      & 0                      & 0                      & 0                       \\ \hline
\textbf{2}                         & 0                      & 0                      & 0                      & 0                       \\ \hline
\textbf{3}                         & 0                      & 0                      & 0                      & 0                       \\ \hline
\textbf{4}                         & 0                      & 0                      & 0                      & 0                       \\ \hline
\textbf{5}                         & 0                      & 0                      & 0                      & 0                       \\ \hline
\textbf{6}                         & 0                      & 0                      & 0                      & 0                       \\ \hline
\textbf{7}                         & 0                      & 0                      & 0                      & 0                       \\ \hline
\end{tabular}
\end{table}

Comentário sobre a tabela.

\end{document}
